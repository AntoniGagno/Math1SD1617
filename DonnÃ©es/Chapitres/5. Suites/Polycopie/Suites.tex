\documentclass[a4paper,11pt]{article}
\usepackage{pdflscape}
\usepackage[utf8]{inputenc}
\usepackage[T1]{fontenc}
%\usepackage{fourier} % math & rm
%\usepackage{amsthm,amsfonts,amsmath,amssymb,textcomp}
\usepackage{pst-all,pstricks-add,pst-eucl}
\everymath{\displaystyle}
\usepackage{fp,ifthen}
%\usepackage{color}
%\usepackage{graphicx}
\usepackage{setspace}
\usepackage{array}
\usepackage{tabularx}
\usepackage{supertabular}
\usepackage{hhline}
\usepackage{variations}
\usepackage{enumerate}
\usepackage{pifont}
\usepackage{framed}
\usepackage[fleqn]{amsmath}
\usepackage{amssymb}
\usepackage[framed]{ntheorem}
\usepackage{multicol}
\usepackage{kpfonts}
\usepackage{manfnt}

%\usepackage[hmargin=2.5cm, vmargin=2.5cm]{geometry}
\usepackage{vmargin}          % Pour fixer les marges du document
\setmarginsrb
{1.5cm} 	%marge gauche
{0.5cm} 	  %marge en haut
{1.5cm}     %marge droite
{0.5cm}   %marge en bas
{1cm} 	%hauteur de l'entête
{0.5cm}   %distance entre l'entête et le texte
{1cm} 	  %hauteur du pied de page
{0.5cm}     %distance entre le texte et le pied de page

\newcommand{\R}{\mathbb{R}}
\newcommand{\N}{\mathbb{N}}
%\newcommand{\D}{\mathbb{D}}
\newcommand{\Z}{\mathbb{Z}}
\newcommand{\Q}{\mathbb{Q}}
\newcommand{\C}{\mathbb{C}}
\newcommand{\e}{\text{e}}
\newcommand{\dx}{\text{d}x}
\newcommand{\vect}[1]{\mathchoice%
  {\overrightarrow{\displaystyle\mathstrut#1\,\,}}%
  {\overrightarrow{\textstyle\mathstrut#1\,\,}}%
  {\overrightarrow{\scriptstyle\mathstrut#1\,\,}}%
  {\overrightarrow{\scriptscriptstyle\mathstrut#1\,\,}}}
\newcommand\arraybslash{\let\\\@arraycr}
\renewcommand{\theenumi}{\textbf{\arabic{enumi}}}
\renewcommand{\labelenumi}{\textbf{\theenumi.}}
\renewcommand{\theenumii}{\textbf{\alph{enumii}}}
\renewcommand{\labelenumii}{\textbf{\theenumii.}}
\renewcommand{\and}{\wedge}

\theoremstyle{break}
\theorembodyfont{\upshape}
\newcounter{enonce}
\newframedtheorem{theorem}[enonce]{Théorème}
\newframedtheorem{proposition}[enonce]{Proposition}
\newframedtheorem{definition}[enonce]{Définition}

\newtheorem{Term}{Terminologie}
\newtheorem{Rq}[enonce]{Remarque}
\newtheorem{exemple}[enonce]{Exemple}
\newtheorem{demonstration}[enonce]{Démonstration}
%\newtheorem{exo}{Exercice}

%\theorembodyfont{\small \sffamily}
%\newtheorem{sol}{solution}

\newenvironment{sol}% 
{\def\FrameCommand{\hspace{0.5cm} {\color{black} \vrule width 1pt} \hspace{-0.7cm}}%
  \framed {\advance\hsize-\width}
  \noindent \small \sffamily  %\underline{Solution :}%\\
}%
{\endframed}

\newrgbcolor{vert}{0 0.4 0}
\newrgbcolor{bistre}{1 .50 .30}
\setlength\tabcolsep{1mm}
\renewcommand\arraystretch{1.3}

\everymath{\displaystyle}
\hyphenpenalty 10000 %supprime toutes les césures
%\setcounter{secnumdepth}{0}
%\newcounter{saveenum}

\usepackage[frenchb]{babel}
%\usepackage{fancyhdr,lastpage}
%\usepackage{fancybox}

%\headheight 15.0 pt
%\fancyhead[L]{Leçon}
%\fancyhead[C]{}
%\fancyhead[R]{Chapitre 1}
%\fancyfoot[L]{{\scriptsize\textsl{Thomas Gire Cité scolaire de Lorgues}}}
%\fancyfoot[C]{\scriptsize\thepage}
%\fancyfoot[C]{\scriptsize\thepage/\pageref{LastPage}}

\title{Suites numériques}
\author{}
\date{}

%\pagestyle{empty}
%\pagestyle{fancy}
\usepackage[np]{numprint}

\renewcommand\arraystretch{1.8}

\newcounter{numero}
\newcommand{\exo}{
  \addtocounter{numero}{1}%
  \textbf{\underline{Exercice \arabic{numero}:}}\quad}

\frenchbsetup{StandardEnumerateEnv=true}
\usepackage{etex}
\usepackage{tikz,tkz-tab}
\usepackage{graphicx}
\graphicspath{ {../images/} }





\begin{document}
  %\setlength{\unitlength}{1mm}
  %\setlength\parindent{0mm}
 
  \maketitle


  
  \section{Définition et mode de génération.}
  
  \subsection{Définition et notations.}
 
  
  \begin{definition}
  Une \textbf{suite} numérique est une fonction définie sur l'ensemble 
  des entiers naturels (sauf eventuellement quelques premiers entiers) à valeurs dans l'ensemble
  des rééls. 
  \end{definition}
  
   \begin{exemple} \label{exsuite}
   \begin{enumerate}
    \item Soit $u:\mathbb{N} \to \mathbb{R}$, $n \mapsto (-1)^n$. On appelle termes de la 
    suite $(u_n)$ les images successives des entiers par $u$.
   On les note $u_n$ au lieu de $u(n)$. $u_0=1$, $u(1)=-1$, $u(2)=1$...
   \item Soit $(v_n)$ la suite définie par la formule $v_n=\frac{1}{n}$. $v_n$ n'est 
   définie qu'à partir de $n=1$.
   \item Soit $(w_n)$ la suite définie par la formule $w_n=\sqrt{n-7}$. $w_n$ 
   n'est définie qu'à partir de $n=7$.
   \end{enumerate}

  \end{exemple}

  \subsection{Définition explicite d'une suite.} 
    
  \begin{definition}
  Une suite numérique peut être définie par la donnée d'une formule 
  \textbf{explicite}
  qui permet de calculer directement chaque terme $u_n$ à l'aide de $n$.
  \end{definition}
  
   \begin{exemple}
   
   Les suites de l'exemple \ref{exsuite}. Pour toute fonction $f:[a,+\infty[$, on peut définir la suite
   $(u_n)_{n \geq a}$ par $u_n=f(n)$.
   
  \end{exemple}
  
   \subsection{Définition d'une suite par récurrence.}
  
   \begin{theorem}
      Une suite numérique peut être définie par la donnée d'un premier terme
      et d'une relation, dite de \textbf{récurrence}, qui permet de calculer 
      un terme à partir du précédent.
   \end{theorem}
   
    \begin{exemple}
    \begin{enumerate}
    \item
    Soit $(u_n)$ la suite définie par récurrence par: $u_0=3$ et pour tout entier n, $u_{n+1}=2u_n-1$.

  $u_1=2u_0-1=2 \times 3-1=5$, $u_2=2 u_1-1=2 \times 5-1=9$. On ne peut pas calculer
  directement $u_n$ à partir de $n$. Par exemple, pour calculer $u_{100}$, il 
  faut calculer tous les termes qui précèdent.
  
  \item Pour toute fonction $g:I \subset \mathbb{R}
  \to I \subset \mathbb{R}$ et $x \in I$, on peut définir la suite $(u_n)$ par $u_0=x$ et $u_{n+1}=g(u_n)$.
  \end{enumerate}
  
   \end{exemple}
   \newpage
  
  \section{Suites arithmétiques.}
  \begin{definition} 
    
   Une suite est \textbf{arithmétique} lorsque l'on passe d'un terme au suivant en ajoutant toujours le même nombre
   appelé la \textbf{raison}.
   
   Autrement dit, une suite $(u_n)_{n \geq p}$ est arithmétique de raison $r$ si et seulement si 
   pour tout entier $n \geq p$, $u_{n+1}=u_n+r$.
   \end{definition}
   
   \begin{exemple}
    
  \begin{enumerate}
   \item La suite des entiers $0,1,2,3,...$ est arithmétique de raison $1$.
   \item La suite des entiers pairs $0,2,4,6,...$ est arithmétique de raison $2$.
   \item la suite des entiers impairs $1,3,5,7,...$ est arithmétique de raison $2$.
   \item la suite des multiples de 5, $0,5,10,15,...$ est arithmétique de raison $5$.
   \item la suite definie par $u_n=7n+4$ pour tout entier $n$. En effet, 
   $u_{n+1}=7(n+1)+4=7n+7+4=7n+4+7=u_n+7$. $u_n$ est arithmétique de raison $7$.
  \end{enumerate}  
   \end{exemple}
   
   \begin{theorem}[Formes explicites d'une suite arithmétique]
    Soit $(u_n)_{n \geq p}$ une suite arithmétique, pour tout couple d'entiers $(n,p)$, $u_n=u_p+(n-p)r$.
   \end{theorem}
   
     \section{Suites géométriques.}
  \begin{definition} 
    
   Une suite est \textbf{géométrique} lorsque l'on passe d'un terme au suivant en multipliant toujours 
   par le même nombre (non nul) appelé la \textbf{raison}.
   
   Autrement dit, une suite $(u_n)_{n \geq p}$ est géométrique de raison $q$ si et seulement si 
   pour tout entier $n \geq p$, $u_{n+1}=u_n \times q$.
   \end{definition}
   
   \begin{exemple}
    
  \begin{enumerate}
   \item La suite des puissances de $2$, $1,2,4,8,16,...$ est géométrique de raison 2.
   \item La suite des puissances de -1, $u_n=(-1)^n$: $1,-1,1,-1,1,...$ de raison $-1$.
   \item la suite $(v_n)_{n \in \mathbb{N}}$ définie pour tout entier $n$ par $v_n=-5 \times 7^n$.
   $v_{n+1}=-5 (7)^{n+1}=-5(7)^n \times 7=v_n \times 7$ et $v_n$ est géométrique de raison $7$.
  \end{enumerate}  
   \end{exemple}
   
   \begin{theorem}[Formes explicites d'une suite géométrique]
    Soit $(u_n)_{n \geq p}$ une suite géométrique, pour tout couple d'entiers $(n,p)$, $u_n=u_p\times q^{n-p}$.
   \end{theorem}

   
   \newpage
   
    \section{Sens de variations.}
  
  \begin{definition}
    
   Soit $(u_n)_{n \geq k}$ une suite numérique.
   
   \begin{itemize}
    \item $u_n$ est \textbf{croissante} si pour tout entier $n \geq k$, $u_{n+1} \geq u_n$.
   \item $u_n$ est \textbf{strictement croissante} si pour tout entier $n \geq k$, $u_{n+1} > u_n$.
   \item $u_n$ est \textbf{décroissante} si pour tout entier $n \geq k$, $u_{n+1} \leq u_n$.
   \item $u_n$ est \textbf{strictement décroissante} si pour tout entier $n \geq k$, $u_{n+1} < u_n$.
   \item $u_n$ est \textbf{constante} si pour tout entier $n \geq k$, $u_{n+1} = u_n$. 
   \end{itemize}
   Une suite croissante ou décroissante est dite \textbf{monotone}.
     
  \end{definition}
  
   \begin{exemple}
    
    \begin{enumerate}
          \item La suite des entiers impairs $u_n=1+2n$ est strictement croissante. En effet, 
     $u_{n+1}-u_n=1+2(n+1)-(1+2n)=2>0$
     \item La suite des inverse $u_n=\frac{1}{n}$ avec $(n>0)$ est strictement décroissante. En effet,
     $u_{n+1}-u_n=\frac{1}{n+1}-\frac{1}{n}=\frac{n-(n+1)}{n(n+1)}=\frac{-1}{n(n+1)<0}$
     
     \item La suite $u_n=(-1)^n$ n'est pas monotone car $u_0=1>-1=u_1<1=u_2$
    \end{enumerate}
 \end{exemple}
 
 \subsection{Sens de variation d'une suite arithmétique.}
 
 \begin{theorem}
  Soit $(u_n)_{n \geq k}$ une suite arithmétique de raison $r$.
  \begin{itemize}
   \item Si $r >0$ alors $(u_n)$ est strictement croissante.
   \item Si $r<0$ alors $(u_n)$ est strictement décroissante.
   \item Si $r=0$ alors $(u_n)$ est constante.
  \end{itemize}
 \end{theorem}
 
 \begin{exemple}
    
    \begin{enumerate}
     \item La suite $(u_n)$ définie par $u_n=1+3n$ est strictement croissante comme 
     c'est une suite arithmétique de raison $3>0$.
     \item La suite $(v_n)$ définie par $v_4=7$ et pour tout entier $n \geq 7$
     , $v_{n+1}=v_n -2$ est strictement décroissante comme suite arithmétique de raison $r=-2<0$.
    \end{enumerate}
 \end{exemple}
 
   \subsection{Sens de variation d'une suite géometrique.}
 
 \begin{theorem}
  Soit $(u_n)_{n \geq k}$ une suite géométrique de raison $q$.
  \begin{itemize}
   \item Si $q > 1$ et $u_0>0$ alors $(u_n)$ est strictement croissante.
   \item Si $q > 1$ et $u_0<0$ alors $(u_n)$ est strictement décroissante.
   \item Si $q = 1$ alors $(u_n)$ est constante.
   \item Si $0 < q < 1$ et $u_0>0$ alors $(u_n)$ est strictement décroissante.
   \item Si $0 < q < 1$ et $u_0<0$ alors $(u_n)$ est strictement croissante.
   \item Si $q < 0$ et $u_0 \neq 0$ alors $(u_n)$ n'est pas monotone.
  \end{itemize}
 \end{theorem}
 
 \begin{exemple}
    
    \begin{enumerate}
     \item La suite $(u_n)_{n \geq 0}$ définie par $u_n=4(\frac{2}{3})^n$ est strictement décroissante comme 
     c'est une suite géométrique de premier terme $4>0$ et de raison $0<\frac{2}{3}<1$.
     \item La suite $(v_n)$ définie par $v_4=-2$ et pour tout entier $n \geq 5$
     , $v_{n+1}=v_n \times 3$ est strictement décroissante comme suite géométrique de premier
     terme $-2<0$ et de raison $3>1$.
     \item La suite $(w_n)_{n \geq 0}$ géométrique de raison $-2$ avec $w_0=3$ n'est pas monotone. En effet,
     $w_0$. En effet, $w_0=3>-6=w_1<w_2=12$.
    \end{enumerate}
 \end{exemple}

 \subsection{Sens de variation d'une suite définie de façon explicite.}
 
  \begin{theorem}
  Soit $f:[k,+\infty[ \to \mathbb{R}$ et $(u_n)_{n \geq k}$ la suite définie par $u_n=f(n)$.
  \begin{itemize}
   \item Si $f$ est (resp. strictement) croissante alors $(u_n)$ est (resp. strictement) croissante.
   \item Si $f$ est (resp. strictement) décroissante alors $(u_n)$ est (resp. strictement) décroissante.
 \end{itemize}
 \end{theorem}
 
 \begin{exemple}
 La suite $(u_n)_{n \geq 1}$ définie par $u_n=\frac{1}{n^2}$ est strictement décroissante comme 
     la fonction $f:]0,+\infty[$,$x \mapsto \frac{1}{x^2}$ est strictement décroissante.
\end{exemple}

\begin{Rq}
 Ne pas confondre avec le cas d'une fonction définie par récurrence.
 \begin{itemize}
  \item La suite $(u_n)$ défine explicitement par $u_n=2n-1$ est strictement croissante comme la fonction 
 $f:x \mapsto 2x-1$ est strictement croissante et $u_n=f(n)$.
 \item Mais la suite $(v_n)$ définie par récurrence par $v_0=0$ et pour tout entier $n$, $v_{n+1}=f(v_n)=2 v_n-1$ 
 n'est pas strictement croissante. En effet, $v_0=0$, $v_1=-1$, $v_2=2(-1)-1=-3$. 
 \end{itemize}

 
 
\end{Rq}



\end{document}