\documentclass[a4paper,11pt]{article}
\usepackage{pdflscape}
\usepackage[utf8]{inputenc}
\usepackage[T1]{fontenc}
%\usepackage{fourier} % math & rm
%\usepackage{amsthm,amsfonts,amsmath,amssymb,textcomp}
\usepackage{pst-all,pstricks-add,pst-eucl}
\everymath{\displaystyle}
\usepackage{fp,ifthen}
%\usepackage{color}
%\usepackage{graphicx}
\usepackage{setspace}
\usepackage{array}
\usepackage{tabularx}
\usepackage{supertabular}
\usepackage{hhline}
\usepackage{variations}
\usepackage{enumerate}
\usepackage{pifont}
\usepackage{framed}
\usepackage[fleqn]{amsmath}
\usepackage{amssymb}
\usepackage[framed]{ntheorem}
\usepackage{multicol}
\usepackage{kpfonts}
\usepackage{manfnt}

%\usepackage[hmargin=2.5cm, vmargin=2.5cm]{geometry}
\usepackage{vmargin}          % Pour fixer les marges du document
\setmarginsrb
{1.5cm} 	%marge gauche
{0.5cm} 	  %marge en haut
{1.5cm}     %marge droite
{0.5cm}   %marge en bas
{1cm} 	%hauteur de l'entête
{0.5cm}   %distance entre l'entête et le texte
{1cm} 	  %hauteur du pied de page
{0.5cm}     %distance entre le texte et le pied de page

\newcommand{\R}{\mathbb{R}}
\newcommand{\N}{\mathbb{N}}
%\newcommand{\D}{\mathbb{D}}
\newcommand{\Z}{\mathbb{Z}}
\newcommand{\Q}{\mathbb{Q}}
\newcommand{\C}{\mathbb{C}}
\newcommand{\e}{\text{e}}
\newcommand{\dx}{\text{d}x}
\newcommand{\vect}[1]{\mathchoice%
  {\overrightarrow{\displaystyle\mathstrut#1\,\,}}%
  {\overrightarrow{\textstyle\mathstrut#1\,\,}}%
  {\overrightarrow{\scriptstyle\mathstrut#1\,\,}}%
  {\overrightarrow{\scriptscriptstyle\mathstrut#1\,\,}}}
\newcommand\arraybslash{\let\\\@arraycr}
\renewcommand{\theenumi}{\textbf{\arabic{enumi}}}
\renewcommand{\labelenumi}{\textbf{\theenumi.}}
\renewcommand{\theenumii}{\textbf{\alph{enumii}}}
\renewcommand{\labelenumii}{\textbf{\theenumii.}}
\renewcommand{\and}{\wedge}

\theoremstyle{break}
\theorembodyfont{\upshape}
\newcounter{enonce}
\newframedtheorem{theorem}[enonce]{Théorème}
\newframedtheorem{proposition}[enonce]{Proposition}
\newframedtheorem{definition}[enonce]{Définition}

\newtheorem{Term}{Terminologie}
\newtheorem{Rq}{Remarque}
\newtheorem{exemple}[enonce]{Exemple}
\newtheorem{demonstration}[enonce]{Démonstration}
%\newtheorem{exo}{Exercice}

%\theorembodyfont{\small \sffamily}
%\newtheorem{sol}{solution}

\newenvironment{sol}% 
{\def\FrameCommand{\hspace{0.5cm} {\color{black} \vrule width 1pt} \hspace{-0.7cm}}%
  \framed {\advance\hsize-\width}
  \noindent \small \sffamily  %\underline{Solution :}%\\
}%
{\endframed}

\newrgbcolor{vert}{0 0.4 0}
\newrgbcolor{bistre}{1 .50 .30}
\setlength\tabcolsep{1mm}
\renewcommand\arraystretch{1.3}

\everymath{\displaystyle}
\hyphenpenalty 10000 %supprime toutes les césures
%\setcounter{secnumdepth}{0}
%\newcounter{saveenum}

\usepackage[frenchb]{babel}
%\usepackage{fancyhdr,lastpage}
%\usepackage{fancybox}

%\headheight 15.0 pt
%\fancyhead[L]{Leçon}
%\fancyhead[C]{}
%\fancyhead[R]{Chapitre 1}
%\fancyfoot[L]{{\scriptsize\textsl{Thomas Gire Cité scolaire de Lorgues}}}
%\fancyfoot[C]{\scriptsize\thepage}
%\fancyfoot[C]{\scriptsize\thepage/\pageref{LastPage}}

\title{Dérivation}
\author{}
\date{}

%\pagestyle{empty}
%\pagestyle{fancy}
\usepackage[np]{numprint}

\renewcommand\arraystretch{1.8}

\newcounter{numero}
\newcommand{\exo}{
  \addtocounter{numero}{1}%
  \textbf{\underline{Exercice \arabic{numero}:}}\quad}

\frenchbsetup{StandardEnumerateEnv=true}
\usepackage{etex}
\usepackage{tikz,tkz-tab}
\usepackage{graphicx}
\graphicspath{ {../images/} }





\begin{document}
  %\setlength{\unitlength}{1mm}
  %\setlength\parindent{0mm}
 
  \maketitle


  
  \section{Tangentes et nombres dérivés.}
  
  \subsection{Tangente à un graphe.}
 
  
  \begin{definition}    
    Soit $f:I \to \R$ une fonction définie sur un intervalle de $\R$. Soit $\mathcal{C}_f$ sa 
    représentation graphique.
    La \textbf{tangente} à $\mathcal{C}_f$ au point $A(a,f(a))$ est la droite passant par $A$ 
    la plus proche de $\mathcal{C}_f$ au voisinage de $a$.
  \end{definition}
  
   \begin{exemple}
   
   La fonction $f:\R \to \R, x \mapsto x^2$ admet une tangente au point $A(1,1)$ dont le coefficient
   directeur vaut $2$.
   
  \end{exemple}

  \subsection{Nombre dérivé.} 
    
  \begin{definition}
  On dit que $f$ est \textbf{dérivable en $\mathbf{a}$} si son graphe $\mathcal{C}_f$ admet une tangente au point 
  $A(a,f(a))$. On appelle alors \textbf{nombre dérivé} et on note $\mathbf{f'(a)}$ le coefficient directeur
  de cette tangente.
  \end{definition}
  
   \begin{exemple}
   
   La fonction $f:\R \to \R, x \mapsto x^2$  est dérivable en $1$ et $f'(1)=2$.
   
  \end{exemple}
  
    \subsection{\'Equation d'une tangente.}
  
   \begin{theorem}
      Soit $f:I \to \R$ une fonction dérivable en $a$. La tangente $T_f(a)$ de $f$ en $A(a,(f(a))$ a
      pour équation $$T_f(a):y=f'(a)(x-a)+f(a)$$
   \end{theorem}
   
    \begin{exemple}
  
  La fonction $f:\R \to \R, x \mapsto x^2$ est dérivable en $1$, $T_f(1)$ a pour équation
  $$T_f(1):y=2(x-1)+1$$
  
   \end{exemple}
  
  
  \begin{definition} 
    
   Soit $f:I \to \R$. On dit que $f$ est \textbf{dérivable sur $\mathbf{I}$} si pour tout réel $a$ dans $I$,
   $f$ est dérivable en $a$. On appelle alors fonction dérivée de $f$ et on note 
   $f':I \to \R, x \mapsto f'(x)$ 
   \end{definition}
   
   \begin{exemple}
    
  La fonction $f:\R \to \R, x \mapsto x^2$ est dérivable sur $\R$ et $f'(x)=2 x$.  
    
   \end{exemple}
   
   \section{Calcul de dérivée.}
   
    \subsection{Fonctions de référence.}
    
    \begin{tabular}{|l|p{2.7cm}|p{3cm}|c|}\hline
     Fonction $f$ & Domaine de définition & Domaine de dérivabilité & Fonction dérivée $f'$ \\ \hline
     Fonction constante : $f(x)=k, k$ réel&  $\R$ & $\R$ & $f'(x)=0$ \\ \hline
     Fonction affine: $f(x)=mx+p$, $m$ et $p$ réels & $\R$ & $\R$ & $f'(x)=m$ \\ \hline
     Fonction puissance: $f(x)=x^n$, n entier naturel & $\R$ & $\R$ & $f'(x)=nx^{n-1}$ \\ \hline
     Fonction inverse: $f(x)=\frac{1}{x}$ & $]-\infty;0[ \cup ]0;+\infty [$ & $]-\infty;0[ \cup ]0;+\infty [$ & $f'(x)=-\frac{1}{x^2}$ \\ \hline
     Fonction racine carrée: $f(x)=\sqrt{x}$& $[0;+\infty[$ & $]0;+\infty[$ & $f'(x)=\frac{1}{2\sqrt{x}}$ \\ \hline
    \end{tabular}
    
    \subsection{Opérations sur les fonctions dérivables.}
  
  \begin{proposition}
  Soient $u$ et $v$ deux fonctions dérivables sur un intervalle $I$ et $\lambda$ un réel. 
  
  \begin{center}
    \begin{tabular}{|c|c|} \hline
   $f$ & $f'$ \\ \hline
   $u+v$ & $u'+v'$ \\ \hline
   $\lambda u$ & $\lambda u'$ \\ \hline
   $uv$ & $u'v+uv'$ \\ \hline
   $\frac{1}{v}$&$-\frac{v'}{v^2}$ \\ \hline
   $\frac{u}{v}$&$\frac{u'v-uv'}{v^2}$ \\ \hline
  \end{tabular}
  \end{center}
  Toutes ces fonctions sont dérivables sur $I$ sauf les fonctions $\frac{1}{v}$ et $\frac{u}{v}$ 
  qui sont dérivables seulement où $v$ ne s'annule pas. 
   
  \end{proposition}
 
 \begin{exemple}
  \begin{enumerate}
  \item $(5)'=0$
  \item $(3x-7)'=3$
  \item $(x^7)'=7x^6$
  \item $(2x^3-x)'=(2x^3)'+(-x)'=2(x^3)'-(x)'=2 \times 3 x^2-1$
  \item $(4x^3+3x^2+5x+14)'=12x^2+6x+5$
  \item $(\sqrt{x}(5x+1))'=(\sqrt{x})'(5x+1)+\sqrt{x}(5x+1)'=\frac{1}{2\sqrt{x}}(5x+1)+5\sqrt{x}$
  \item $(\frac{1}{x^3+x})'=-\frac{(x^3+x)'}{(x^3+x)^2}=-\frac{3x^2+1}{(x^3+x)^2}$
  \item $\frac{x^2+3x+7}{7x+1}=\frac{(x^2+3x+7)'(7x+1)-(x^2+3x+7)(7x+1)'}{(7x+1)^2}=
  \frac{(2x+3)(7x+1)-(x^2+3x+7)\times 7}{(7x+1)^2}$
  \end{enumerate}

\end{exemple}

\newpage

\section{Signe de la dérivée et sens de variation.}
  
  \begin{theorem}
    
   Soit $f$ une fonction dérivable sur un intervalle $I$.
   
   Pour tout $x$ de $I$, $f'(x) \geq 0 \Leftrightarrow$ $f$ est croissante sur $I$.
   
   Pour tout $x$ de $I$, $f'(x) \leq 0 \Leftrightarrow$ $f$ est décroissante sur $I$.
   
   Pour tout $x$ de $I$, $f'(x)=0 \Leftrightarrow$ $f$ est constante sur $I$.
   
   Pour tout $x$ de $I$ sauf un nombre fini $f'(x) >0
   \Leftrightarrow$ $f$ est strictement croissante sur $I$.
   
   Pour tout $x$ de $I$ sauf un nombre fini $f'(x) <0
   \Leftrightarrow$ $f$ est strictement décroissante sur $I$.
  \end{theorem}
  
   \begin{exemple}
  
    Soit $f$ la fonction définie sur l'intervalle $[-3;3]$ par $f(x)=-2x^3 -1,5x^2+18 x + 26$
    
    \begin{enumerate}
     \item \'Etudier les variations de la fonction $f$ sur l'intervalle $[-3;3]$.
     \item En déduire les extremums de la fonction $f$ et préciser en quelles valeurs elles sont
     atteintes.
    \end{enumerate}
    
    \begin{enumerate}
     \item On dérive la fonction $f$, $f'(x)=-6x^2-3x+18$.
     
     $f'(x)$ est une fonction trinôme du second degré, avec $a=-6$, $b=-3$ et $c=18$.
     
     $\Delta=(-3)^2-4(-6)(18)=441=21^2$, $x_1=\frac{3-21}{-12}=\frac{3}{2}$ et 
     $x_2=\frac{3+21}{-12}=-2$.
     
     On dresse alors le tableau de signe de $f$ et on déduit les variations de $f$:
      
      \begin{tikzpicture}
\tkzTabInit{$x$/1,$f'(x)$/1,$f(x)$/2}{$-3$,$-2$,$\frac{3}{2}$,$3$}
\tkzTabLine{,-,z,+,z,-}
\tkzTabVar{+/12.5,-/0,+/42.875,-/12.5}
\end{tikzpicture}
     \item Sur $(-3;3]$, 0 est le minimum de la fonction $f$ atteint pour $x=-2$. 42.875 est le maximum
     de la fonction $f$ atteint pour $x=\frac{3}{2}$.
    \end{enumerate}
 \end{exemple}
\end{document}