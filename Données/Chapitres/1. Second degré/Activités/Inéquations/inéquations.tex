\documentclass{beamer}
\usepackage[utf8]{inputenc}

%\usetheme{Warsaw}
%\usetheme{Hannover}
\usetheme{Berkeley}
%\usecolortheme{lily}
\setbeamertemplate{theorems}[numbered] 
%\setbeamertemplate{theorems}[ams style]
\date{}

%\theoremstyle{plain}

\usepackage{lmodern}
\usepackage[T1]{fontenc}
\usepackage[utf8]{inputenc}
\usepackage[french]{babel}
\usepackage{tikz,tkz-tab}

\uselanguage{French}
\languagepath{French}

\newtheorem{proposition}[theorem]{\translate{Proposition}}
\newtheorem{exercise}[theorem]{\translate{Exercise}}
\newtheorem{methode}[theorem]{{Méthode}}

\title{Inéquations du second degré.}
\subtitle{Exercices}

\begin{document}
  
  \begin{frame}
    
    \titlepage
    
  \end{frame}
  
  
    
  \section{Signe d'un trinôme.}
  \subsection{Réaliser un tableau de signe}
  \begin{frame} 
    
  \begin{exercise}
    \'Etudier le signe des fonctions définies par les expressions suivantes:
    \begin{itemize}
     \item $a(x)=2x+3$.
     \item $b(x)=(x+2)(x-5)$
     \item $c(x)=\frac{3x+2}{5x-1}$
     \item $d(x)=x^2-1$
   \end{itemize}    
  \end{exercise}
  
  \end{frame}
  
  \begin{frame}
    \begin{methode}
    \'Etude du signe de fonctions :
    \begin{itemize}
      \item $a(x)=2x+3$. Je reconnais \uncover<2,3,4,5,6,7,8,9,10,11,12,13>{une fonction affine}, 
      c'est à dire de la forme \uncover<3,4,5,6,7,8,9,10,11,12,13>{$ax+b$}.
      Je déduis du \uncover<4,5,6,7,8,9,10,11,12,13>{signe de a} les variations de $f$. Je peux alors déterminer 
      \uncover<5,6,7,8,9,10,11,12,13>{le signe} de la fonction affine autour de sa racine
      \uncover<6,7,8,9,10,11,12,13>{$-\frac{b}{a}$}.
      \item $b(x)=(x+2)(x-5)$ est un 
      \uncover<7,8,9,10,11,12,13>{produit de fonctions affines} dont je sais
      étudier le signe comme pour $a(x)$. J'utilise 
      \uncover<8,9,10,11,12,13>{la règle des signes} 
      pour réaliser le tableau de signes.
     \item $c(x)=\frac{3x+2}{5x-1}$ est un \uncover<9,10,11,12,13>{quotient} 
     de fonctions dont je sais étudier le signe. J'utilise à nouveau 
     \uncover<10,11,12,13>{la règle des signes} et
     je n'oublie pas de représenter \uncover<11,12,13>{la valeur interdite} 
     par une double barre .
     \item $d(x)=x^2-1$. Je reconnais \uncover<12,13>{une identité remarquable} qui permet
     de \uncover<13>{factoriser}. Je termine comme pour $b(x)$.
     \item $e(x)=x^2+2x+2$
     \item $f(x)=x^2-4x+3$
     \item $g(x)=3x^2-18x+27$
     \item $h(x)=-2x^2-2x+12=-2(x^2+x-6)=2(x+3)(x-2)$
     \item $i(x)=\frac{3x^2+3x-6}{2x-2}$
   \end{itemize}    
  \end{methode}
  \end{frame}
  
  \subsection{Résoudre une inéquation du second degré}
  
  \begin{frame}
  \begin{exercise}
    Résoudre les inéquations suivantes:
    \begin{itemize}
     \item $x^2+2x+2>0$
     \item $x^2-4x+3<0$
   \end{itemize}    
  \end{exercise} 
  \end{frame}
  
  \begin{frame}
  \begin{methode}
    \begin{itemize}
     \item $x^2+2x+2>0$. J'introduis le trinôme $j(x)=x^2+2x+2$. Je calcule son 
     \uncover<2,3,4,5,6,7,8,9>{discriminant}. 
     Il est strictement négatif. J'en déduis 
     \uncover<3,4,5,6,7,8,9>{qu'il n'y a pas de racine} et que le signe de
     $j(x)$ est \uncover<4,5,6,7,8,9>{constant}. Celui de $f(0)=2>0$ par exemple. Je pense
     à exhiber l'ensemble \uncover<5,6,7,8,9>{$S$ des solutions}
     pour conclure.
     
     Méthode alternative: J'utilise \uncover<6,7,8,9>{la forme canonique}: 
     $j(x)=(x^2+2x+1)+1=(x+1)^2+1>0$
     \item $x^2-4 x+3<0$ J'introduis \uncover<7,8,9>{le trinôme
       $k(x)=x^2-4x+3$}. Je calcule son discriminant.
     Il est strictement positif. Je calcule \uncover<8,9>{les racines}
     et je \uncover<9>{factorise}
     $k(x)=(x-1)(x-3)$. Je réalise le tableau de signe et je conclus en donnant
     l'ensemble des solutions.
   \end{itemize}
 \end{methode}
\end{frame}

  \begin{frame}
  \begin{exercise}
    Résoudre sur $\mathbb{R}$ l'inéquation:
    \begin{itemize}
     \item $3x^2-18x+31<4$
   \end{itemize}    
  \end{exercise} 
  \end{frame}

  
  \begin{frame}
  \begin{methode}
    \begin{itemize}
      \item $3x^2-18x+31<4$ \uncover<2,3,4,5,6,7,8,9>{En transposant}, 
      je ramène la résolution de l'inéquation à l'étude 
      du signe du trinôme 
     \uncover<3,4,5,6,7,8,9>{$l(x)=3x^2-18x+27$}. Je 
      factorise les coefficients pour faciliter mes calculs: 
      $l(x)=3\uncover<4,5,6,7,8,9>{(x^2-6x+9)}$. Je calcule 
      \uncover<5,6,7,8,9>{le discriminant}. Il est nul. 
      Le signe du trinôme est \uncover<6,7,8,9>{constant}, par exemple 
      celui de $f(0)=c=27>0$. Je pense à 
      exclure \uncover<7,8,9>{la racine double} des solutions 
      comme l'inégalité était \uncover<8,9>{stricte}. 
     
     Méthode alternative: Je reconnais une \uncover<9>{identité
     remarquable}: $l(x)=3(x-3)^2$.
     
   \end{itemize}    
  \end{methode} 
  \end{frame}
  
   \begin{frame}
  \begin{exercise}
    Résoudre sur $\mathbb{R}$ les inéquations suivantes:
    \begin{itemize}
      \item $-2x^2+6>2x-6$
     \item $\frac{3x^2+11x+2}{2x-2}>4$
   \end{itemize}    
  \end{exercise} 
  \end{frame}
  
  \begin{frame}
  \begin{methode}
    \begin{itemize}
      \item $-2x^2+5>2x-7$ Je me ramène à une étude de signe en transposant.
      $-2x^2-2x+12=-2(x^2-x-6)$. Je calcule \uncover<2,3,4,5,6,7,8>{le discriminant}. Il
      est \uncover<3,4,5,6,7,8>{strictement positif}. Je calcule \uncover<4,5,6,7,8>{les racines}.
      Je \uncover<5,6,7,8> {factorise} le trinôme. Je réalise le \uncover<6,7,8>{tableau de signe} et
      je conclus.
     
      \item $\frac{3x^2+11x+2}{2x-2}>4$. Je transpose et je 
      \uncover<7,8>{réduis au même dénominateur}.
    $\frac{3(x^2+x-2)}{2x-2}>0$. Je factorise le trinôme 
    $x^2+x-2=(x-1)(x+2)$, je construis un tableau de signe et je conclus en pensant
    \uncover<8>{aux valeurs interdites}.
   \end{itemize}    
  \end{methode} 
  \end{frame}


  
\end{document}



  
  
  
  

\RequirePackage{framed} %  ou usepackage. permet d'encadrer les theo.

\RequirePackage[amsmath,thmmarks,hyperref,framed]{ntheorem}


\let\remarque\undefined

\theoremstyle{break}                  % passage à la ligne
\theorembodyfont{\itshape}     % fonte
\newtheorem{theoreme}{Th\'eor\`eme}
\newtheorem{proposition}{Proposition}
\newtheorem{propriete}{Propri\'et\'e}
\newtheorem{lemme}{Lemme}
\newtheorem{corollaire}{Corollaire}
\newtheorem{axiome}{Axiome}
\newframedtheorem{Theoreme}[theoreme]{Th\'eor\`eme} % theo. encadré
\newframedtheorem{Axiome}[axiome]{Axiome}
\newframedtheorem{Propriete}[propriete]{Propri\'et\'e}
%\newframedtheorem{Corollaire}[corollaire]{Corollaire}

\theorembodyfont{\upshape} % nouvelle fonte
\newtheorem{exemple}{Exemple}
\newtheorem{remarque}{Remarque}
\newtheorem{definition}{D\'efinition}
\newtheorem{convention}{Convention}
\newtheorem{note}{Note}
\newtheorem{representation}{Repr\'esentation graphique}
\newframedtheorem{Exemple}[exemple]{Exemple}


\theoremstyle{nonumberbreak} % pas de numerotation (de mémoire...)
\theoremheaderfont{\scshape}
\theoremsymbol{\ensuremath\square} % symbole en fin de theo.
\newtheorem{demonstration}{D\'emonstration}
\newtheorem{preuve}{D\'emonstration}